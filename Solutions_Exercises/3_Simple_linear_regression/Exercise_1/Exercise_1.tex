\documentclass{article}
\usepackage{amsmath, amssymb}

\title{Solution to Exercise 1}
\author{}
\date{}

\begin{document}

\begin{enumerate}
    \item What is the expected height when  $x_i = \bar{x}$?
    \item What is the expected height when $ x_i $ changes by 1 unit?
\end{enumerate}
\vspace{1cm}
\noindent
1) Expected height when \( x_i = \bar{x} \)\\
\\
Since $x_i - \bar{x} = 0$, it follows that\\
\\
$\mu_i = \alpha$,\\
\\
which can be interpreted as height of a person who has average weight.\\
\\
2) If $x_i$ increases by one unit, the weight is $1kg$ above the average weight,
and the expected height changes by $\beta cm$, which means an increase or decrease
depending on the sign of $\beta$ (in our case, $\beta >0$).\\
\\
In general one could say that for two people that differ in weight by $1kg$, the expected
height difference is $\beta kg$.

\end{document}